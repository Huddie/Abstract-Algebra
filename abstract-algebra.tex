\documentclass{article}

% Packages
\usepackage[utf8]{inputenc}
\usepackage{amsthm}
\usepackage{amsmath}
\usepackage{amssymb}
\usepackage{romannum}
\usepackage{textcomp}
\usepackage{titlesec}
\usepackage{array}
\usepackage{booktabs}

% Counters
\newcounter{example}
\newcounter{claim}
\newcounter{solution}
\setcounter{secnumdepth}{4}

% Styles
\theoremstyle{definition}
\newtheorem{definition}{Definition}[section]

\theoremstyle{claim}
\newtheorem{claim}{Claim}[section]
 
\theoremstyle{remark}
\newtheorem*{remark}{Remark}

\theoremstyle{theorem}
\newtheorem{theorem}{Theorem}

\titleformat{\paragraph}
{\normalfont\normalsize\bfseries}{\theparagraph}{1em}{}
\titlespacing*{\paragraph}
{0pt}{3.25ex plus 1ex minus .2ex}{1.5ex plus .2ex}


% Commands

\newcommand\Example{%
  \stepcounter{example}%
  \textbf{Example \theexample.}~%
  \setcounter{solution}{0}%
}

% The Solution is used when only
% 1 solution exists
\newcommand\TheSolution{%
  \textbf{Solution:}\\%
}

% If more than 1 solution exists to a
% specific problem/ example use 
% a solution.
\newcommand\ASolution{%
  \stepcounter{solution}%
  \textbf{Solution \thesolution:}\\%
}


\title{Abstract Structures 333}

\begin{document}

\maketitle

% Flush left to keep starting point
% of all lines flush (Design pref.)
\begin{flushleft}

\section{Equivalence Relations}
\begin{definition}{Equivalence Relation}

An equivalence relation, denoted by the symbol \char`\~ , on a set $\Huge S$ is a set $\Huge R$
\footnote{Need not be unique}
of ordered pairs (a, b) $\in$ S x S such that:
\begin{enumerate}
  \item (a, a) $\in$ R $\forall$ a $\in$ S
  \item (a, b) $\in$ R implies (b, a) $\in$ R $\forall$ a, b $\in$ S
  \item (a, b), (b, c) $\in$ R implies (a, c) $\in$ R $\forall$ a, b, c $\in$ S
\end{enumerate}
\end{definition}

We are concerned with \textbf{what partition is R imposing on S}
\newline
\newline

\Example Define $\mathbb{Z}$ in the following way\\ $\newline$ Fix n $\in$ $\mathbb{Z+}$ $\newline$\\ We say a is congruent $\equiv$ to b (mod n) or $a \equiv \text{b (mod n)}$ iff n $\mid$ (a-b)

Show that the example above is an equivalence relation
\newline \newline
\TheSolution 
\begin{proof}{}
We must prove the following 3 properties
\begin{enumerate}
  \item Reflexive [ ($\romannum{1}$) in the def of ER]
  \begin{itemize}
     \item Thought of as: An element a is always related (\char`\~) to itself.
   \end{itemize}
   We are trying to prove that $a \equiv\text{a (mod n)}$. We can start by rewriting this congruence as $n \mid (a-a)$ by def of congruence. This leaves us with $n \mid (0)$ which is true for all $n > 0$. Since n be def is fixed in $\mathbb{Z+}$, this congruence will always hold.
  \item Symmetric [ ($\romannum{2}$) in the def of ER]
    \begin{itemize}
     \item Thought of as: Given $(a, b)$ is valid, we can show $(b, c)$ is valid.
    \end{itemize}
   Since we are given $(a, b)$ is valid, we can write $a \equiv \text{b (mod n)}$ or $n \mid (a-b)$. We must show that $b \equiv a (mod n)$ or $n \mid (b-a)$.
   We can rewrite $n \mid (b-a)$ as $-1 * n \mid (a-b)$. Since we know $n \mid (a-b)$ from out given, we know that this division holds true and therefore $n \mid (b-a)$ as well. 
  \item Transitive [ ($\romannum{3}$) in the def of ER]
    \begin{itemize}
     \item Thought of as: Given $a \char`\~ b$ and $b \char`\~ c$ we must show $a \char`\~ c$.
   \end{itemize}
   We can write the congruence as 2 linear equation.
   \begin{itemize}
     \item $nk = a - b$
     \item $nl = a - c$
   \end{itemize}
   Rearranging we get: $n(k+l) = a - c$ which can be rewritten as $n \mid (a-c)$
\end{enumerate}
\end{proof}
Now that we have proved that a congruence is an $ER$ on $S = \mathbb{Z}$ we would like to see what affect it has on $\mathbb{Z}$. $ie:$ What is $a \char`\~ b$ / what partition does it impose.
\newline
\newline
\Example Take $n = 5$, given the following values for $a$ which values in $\mathbb{Z}$ satisfy the congruence $a \equiv \text{b (mod n)}$ and is the resulting set equal to $\mathbb{Z}$?
\begin{itemize}
     \item $a = 0$
     \item $a = 1$
     \item $a = 2$
\end{itemize}
\TheSolution 
\begin{itemize}
     \item $\{\pm{0}, \pm{5}, \pm{10}\dots\}$
     \item $\{\pm{1}, \pm{6}, \pm{5k+1}\dots\}$
     \item $\{\pm{2}, \pm{7}, \pm{5k+2}\dots\}$
\end{itemize}
No sets equal $\mathbb{Z}$
\newline
\newline
We can see that it appears that a congruence will always split the set $\mathbb{Z}$ into n partitions.
\newline
\begin{definition}{Partition of a set}

A partition of a set S is a collection of \textbf{non-empty, disjoint} subsets $\{s_{0}, s_{1}, \dots\}$ such that (st) $\bigcup\limits_{i=1}^{\infty} S_{i} = S$

\end{definition}
\begin{theorem}{The equivalence classes of a set S under \char`\~ form a partition of S}
\begin{proof}{
We need to show that given \char`\~, we are left with a collection of \textbf{disjoint} subsets who's union is S.

Let a \char`\~ S. We know a is in  its own set because a \char`\~ a. So $\forall$ a $\in$ S the set containing a is \textbf{non-empty}. If we do this for all a $\in$ S then the union of those sets is S. So we need only show that these sets are \textbf{disjoint}.
}
\end{proof}
\end{theorem}

\Example Let S = $\mathbb{Z}x\mathbb{Z}$ [(a, b) a,b $\in$ $\mathbb{Z}$]
Define $\char`\~$ on S by (a, b) $\char`\~$ (c, d) iff ad = bc
\begin{enumerate}
  \item Prove $\char`\~$ is an ER
  \item What partition of $\mathbb{R}x\mathbb{R}$ does this impose
\end{enumerate}
\TheSolution 
\begin{proof}
If ER, 3 properties must hold:
\begin{enumerate}
  \item Reflexive: (a, b) $\char`\~$ (a, b) $\Longrightarrow$ ab = ab which is true.
  \item Symmetric: Given (a, b) $\char`\~$ (c, d) we can show (c, d) $\char`\~$ (a, b). (a, b) $\char`\~$ (c, d) $\Longrightarrow$ ad = bc, (c, d) $\char`\~$ (a, b) $\Longrightarrow$ cb = da. Since we are in the realm of $\mathbb{R}$ we can rearrange to bc = ad which is equal to ad = bc.
  \item Transitive: We must show that if (a, b) $\char`\~$ (c, d) and (c, d) $\char`\~$ (e, f) then (a, b) $\char`\~$ (e, f). We can write it as follows ad = bc and cf = de  the af = be.
  We can write 
  \begin{gather*} 
  adcf = bcde   \\
  ace = bce     \\
  af = be       \\
  \end{gather*} 
\end{enumerate}
To find the partition we may start by plugging in random values.\\
\begin{gather*} 
(1, 1)  \\
1d = 1c \\
d = c   \\
= \{(1, 1), (2, 2), \dots, (n, n)\}
\end{gather*}\\
\begin{gather*} 
(1, 2)\\
d = 2c\\
= \{(1, 2), (2, 4), \dots, (n, 2n)\}\\
\vdots\\
\infty
\end{gather*} 
This partition forms all rational numbers. The first set represents $\frac{1}{1}$ or $1$, the second represents $\frac{1}{2}$ $\dots$ $\infty$
\end{proof}
\Example Let S = $\mathbb{R}-\{0\}$\\Define a $\char`\~ b \leftrightarrow ab > 0$\\What partition does that make on $\mathbb{R}$\newline\\
\TheSolution By plugging in we see we get 2 sets.
\begin{enumerate}
  \item \{1, 2, \dots, n\} = All positive integers
  \item \{-n, -n-1, \dots, -1\} = All negative integers
\end{enumerate}
\begin{theorem}{Division Algorithm}\\
Let D $\in$ $\mathbb{Z+}$, a $\in$ $\mathbb{Z}$, $\exists !$q, r s.t. $a = dq + r$ when $0 < r \leq d$
\end{theorem}
\Example a = 100, d = 7\newline\\
\TheSolution
\begin{gather*} 
100 = 7q + r = 7(14) + 2\\
7 = 2q + r = 2(3) + 1   \\
2 = 1q + r = 1(2) + 0   \\
\end{gather*} 
So, 1 would be the GCD.

\section{Chapter 1: Groups}
\begin{definition}{Binary Operation}
We define a binary operation on set S is a function from SxS $\longrightarrow$ S\\
ie: Takes a pair of elements in S and sends them to another element in S
\end{definition}
\Example Let S = $\mathbb{Z}$, with bin-op (+)\\
a + b = c   \\
3 + 5 = 8   \\
3 $\in$ $\mathbb{Z}$, 5 $\in$ $\mathbb{Z}$, 8 $\in$ $\mathbb{Z}$.

\begin{definition}{Let S be a set w/ bin-op $*$\footnote{$*$ denotes any bin-op}}\\
If $\forall$ a, b $\in$ S, a + b $\in$ S we say S is closed (under $*$)
\end{definition}
\Example                            \\
($M_{22}$, $\cdot$) is closed       \\
($\mathbb{R}$, $\div$) is not closed\\
\begin{definition}{Let G be a set closed under bin-op *}
G is a group if the following hold:
\begin{enumerate}
  \item Associative: $\forall$ a, b, c $\in$ G we have $(a * b) * c = a * (b * c)$
  \item $\exists$ an Identity in G s.t. $\forall$ a $\in$ G we have $(e * a) = (a * e) = a$
  \item ($\forall$ a $\in$ G)$\exists a^{-1}$ s.t. $a * a\prime = a^{-1} * a = e$
\end{enumerate}
\end{definition}
\Example $\mathbb{Z}_{n}$ = the group \{0, 1, 2, $\dots$, n-1\} under addition mod n.\\
What is addition mod n?\newline \\
For a, b $\in \mathbb{Z}_{n}$:  \\
if $a + b < n$, $a + b = a + b$ \\
if $a + b \geq n$, $a + b = a + b - n$\newline\\

\begin{enumerate}
  \item Associative: We are dealing with integers so associativity holds.
  \item $\exists$ an Identity: The identity is 0 ($e = 0$) 
  \item ($\forall$ a $\in$ G)$\exists a^{-1}$: The inverse is $n - a$
\end{enumerate}

\paragraph{Common Groups}
\begin{itemize}
  \item ($\mathbb{Z}$, +)
  \item ($\mathbb{R}$, +)
  \item ($\mathbb{C}$, +)
  \item ($GL_{2R}$, *)
\end{itemize}

\begin{proof} of  ($GL_{2R}$, *)\\
  We know from linear algebra that $det(AB) = det(A)det(B)$\\
  We also know that the identity 2x2 matrix is 
  \[
M_{2x2}=
  \begin{bmatrix}
    1 & 0  \\
    0 & 1
  \end{bmatrix}
\]\\
Additionally, we are able to inherit associativity from general matrices. 
This leaves inverse.\\
We prove inverse as follows:
\begin{gather*} 
A^{-1} = \frac{1}{det(A)} 
\[
    \begin{bmatrix}
      d & -b  \\
      -c & a
    \end{bmatrix}
\]
\end{gather*} 
\end{proof}

\begin{definition}{Order of a  group}\\
The order of a group, G, denoted $| G |$ is the number of elements in G as a set\\
If set G has a finite number of elements we say G is a finite group. If G has an infinite number of elements we say G is an infinite group.
\end{definition}

\begin{definition}{Abelian Groups}\\
If a group is commutative, we say it is Abelian. If not, we say its not-Abelian
\end{definition}

\begin{definition}{Cayley Table}\\
A cayley table is a way to describe the structure of a finite group.\\
Properties that may be derived from a cayley table are:
\begin{itemize}
  \item If the table is reflect-able, the group is Abelian 
  \item Every element appears in each row/column
  \item Easily find the identity (The row/column which entries is equal to the input)
\end{itemize}
\end{definition}

\Example Write the Cayley Table for $\mathbb{Z}_{3}$\\
 \begin{center}
\setlength\extrarowheight{3pt}
\begin{tabular}{c | c c c c c}
    $\mathbb{Z}_{3}$ & 0 & 1 & 2  \\
    \cline{1-4}
    0 & 0 & 1 & 2  \\
    1 & 1 & 2 & 0  \\
    2 & 2 & 0 & 1  \\
\end{tabular}
\end{center}

\Example Write the Cayley Table for $|G|=3$\\
 \begin{center}

\setlength\extrarowheight{3pt}
\begin{tabular}{c | c c c c c}
    $|G|=3$ & e & a & b  \\
    \cline{1-4}
    e & e & a & b  \\
    a & a & b & e  \\
    b & b & e & a  \\
\end{tabular}
\end{center}
Notice that this the second table above was forced. Meaning, no other configuration of $e,a,b$ could have been entered into the table and the table maintain all group properties.\\\\
We see from this that there is only 1 group with order 3. Even though we may label that group with different elements, the underlying groups are all the same.

\begin{claim}{}
$\exists$! 2 groups of order 4 ($|G|=4$)
\end{claim}

\begin{center}
\setlength\extrarowheight{3pt}
\begin{tabular}{c | c c c c c c}
    $\mathbb{Z}_{4}$ & 0 & 1 & 2 & 3  \\
    \cline{1-5}
    0 & 0 & 1 & 2 & 3  \\
    1 & 1 & 2 & 3 & 0 	\\
    2 & 2 & 3 & 0 & 1	\\
    3 & 3 & 0 & 1 & 2	\\
\end{tabular}
\end{center}

\begin{center}
\setlength\extrarowheight{3pt}
\begin{tabular}{c | c c c c c c}
    $\mathbb{Z}_{2x2}$ &(0,0) & (0,1) & (1,0) & (1,1)  \\
    \cline{1-5}
    (0,0) & (0,0) & (0,1) & (1,0) & (1,1)  	\\
    (0,1) & (0,1) & (0,0) & (1,1) & (1,0) 	\\
    (1,0) & (1,0) & (1,1) & (0,0) & (0,1)		\\
    (1,1) & (1,1) & (1,0) & (0,1) & (0,0)		\\
\end{tabular}
\end{center}\\
Any other groups of order 4 will have a bijection to either $\mathbb{Z}_{4}$ or $\mathbb{Z}_{2x2}$\\
Here is an example of one of those:

\Example Let G = \{1, -1, i, -i\} under *\\

\begin{center}
\setlength\extrarowheight{3pt}
\begin{tabular}{c | c c c c c c}
    $\mathbb{G}_{4}$ & 1 & -1 & i & -i  \\
    \cline{1-5}
    $1$&$1$&$-1$&$ i$&$-i$  	l\\
    $-1$&$-1$&$1$&$-i$&$i$	\\
    $i$&$i$&$-i$&$-1$& $1$	\\
    $-i$&$-i$&$i$&$1$&$-1$	\\
\end{tabular}
\end{center}

\begin{claim}{}
If groups are structurally identical, then, you can find a bijection $\phi$(G_{1}) =  G_{2}
\end{claim}
\begin{definition}{Order of an element}\\
Let $G$ be a group with g $\in$ $G$. The order of g (referred to as the order of the element) is the smallest positive integer n s.t. $g^{n} = e$ where e is the identify element of $G$. 
\begin{definition}{Cyclic Group}\\
Let $G$ be a group with order $n$. We say $G$ is cyclic if $\exists g \in G$ s.t $|g|=n$

\begin{theorem}{Let a $\in  \mathbb{Z}_{n}$, then $|a| = \frac{n}{(a,n)}$\end{theorem}

\begin{proof}{$|a|$ is the smallest positive integer $k$ s.t. $ka  \equiv 0 (n)$\\i.e $ln-ka = 0$\\Solve for k using linear diophantine equation}

\begin{claim}{\mathbb{Z}_{n}$ is a cyclic group} \end{claim}

\begin{proof}{We know  $\mathbb{Z}_{n}$ is cyclic if $\exists$ some a s.t. $|a| = n$ where $n = |G|$ by the proof above (Them. 3), $|a| = \frac{n}{(a,n)}$ $\therefore$ $(a, n) = 1$. To prove that $\mathbb{Z}_{n}$ is cyclic we must show that $\exists$ an a such that $(a, n) = 1$\\We will choose $n-1$ as our $a$ giving us $(n-1, n) = 1$ which is always true}
\end{proof}

\begin{definition}{Group Generator}\\
Let $G$ be a cyclic group of order $n$.\\
If $a\inG$ has order $n$, we call $a$ a generator of $G$ and we write $<a> = G$ . We say "the group generated by a"
\end{definition}

\end{flushleft}
\end{document}
