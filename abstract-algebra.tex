\documentclass{article}
%%%%%%%%%%%%%%%%%%%%%%%%%%%%%%%%%%%%%%%%%
% The Legrand Orange Book
% Structural Definitions File
% Version 2.1 (26/09/2018)
%
% Original author:
% Mathias Legrand (legrand.mathias@gmail.com) with modifications by:
% Vel (vel@latextemplates.com)
% 
% This file was downloaded from:
% http://www.LaTeXTemplates.com
%
% License:
% CC BY-NC-SA 3.0 (http://creativecommons.org/licenses/by-nc-sa/3.0/)
%
%%%%%%%%%%%%%%%%%%%%%%%%%%%%%%%%%%%%%%%%%

%----------------------------------------------------------------------------------------
%	VARIOUS REQUIRED PACKAGES AND CONFIGURATIONS
%----------------------------------------------------------------------------------------

\usepackage{graphicx} % Required for including pictures
\graphicspath{{Pictures/}} % Specifies the directory where pictures are stored

\usepackage{lipsum} % Inserts dummy text

\usepackage{tikz} % Required for drawing custom shapes

\usepackage[english]{babel} % English language/hyphenation

\usepackage{enumitem} % Customize lists
\setlist{nolistsep} % Reduce spacing between bullet points and numbered lists

\usepackage{booktabs} % Required for nicer horizontal rules in tables

\usepackage{xcolor} % Required for specifying colors by name
\definecolor{ocre}{RGB}{102,102,230} % Define the orange color used for highlighting throughout the book

%----------------------------------------------------------------------------------------
%	MARGINS
%----------------------------------------------------------------------------------------

% \usepackage{geometry} % Required for adjusting page dimensions and margins

\geometry{
	paper=a4paper, % Paper size, change to letterpaper for US letter size
	top=3cm, % Top margin
	bottom=3cm, % Bottom margin
	left=3cm, % Left margin
	right=3cm, % Right margin
	headheight=14pt, % Header height
	footskip=1.4cm, % Space from the bottom margin to the baseline of the footer
	headsep=10pt, % Space from the top margin to the baseline of the header
	%showframe, % Uncomment to show how the type block is set on the page
}

%----------------------------------------------------------------------------------------
%	FONTS
%----------------------------------------------------------------------------------------

\usepackage{avant} % Use the Avantgarde font for headings
%\usepackage{times} % Use the Times font for headings
\usepackage{mathptmx} % Use the Adobe Times Roman as the default text font together with math symbols from the Sym­bol, Chancery and Com­puter Modern fonts

\usepackage{microtype} % Slightly tweak font spacing for aesthetics
\usepackage[utf8]{inputenc} % Required for including letters with accents
\usepackage[T1]{fontenc} % Use 8-bit encoding that has 256 glyphs

%----------------------------------------------------------------------------------------
%	BIBLIOGRAPHY AND INDEX
%----------------------------------------------------------------------------------------

\usepackage[style=numeric,citestyle=numeric,sorting=nyt,sortcites=true,autopunct=true,babel=hyphen,hyperref=true,abbreviate=false,backref=true,backend=biber]{biblatex}
\addbibresource{bibliography.bib} % BibTeX bibliography file
\defbibheading{bibempty}{}

\usepackage{calc} % For simpler calculation - used for spacing the index letter headings correctly
\usepackage{makeidx} % Required to make an index
\makeindex % Tells LaTeX to create the files required for indexing

%----------------------------------------------------------------------------------------
%	MAIN TABLE OF CONTENTS
%----------------------------------------------------------------------------------------

\usepackage{titletoc} % Required for manipulating the table of contents

\contentsmargin{0cm} % Removes the default margin

% Part text styling (this is mostly taken care of in the PART HEADINGS section of this file)
\titlecontents{part}
	[0cm] % Left indentation
	{\addvspace{20pt}\bfseries} % Spacing and font options for parts
	{}
	{}
	{}

% Chapter text styling
\titlecontents{chapter}
	[1.25cm] % Left indentation
	{\addvspace{12pt}\large\sffamily\bfseries} % Spacing and font options for chapters
	{\color{ocre!60}\contentslabel[\Large\thecontentslabel]{1.25cm}\color{ocre}} % Formatting of numbered sections of this type
	{\color{ocre}} % Formatting of numberless sections of this type
	{\color{ocre!60}\normalsize\;\titlerule*[.5pc]{.}\;\thecontentspage} % Formatting of the filler to the right of the heading and the page number

% Section text styling
\titlecontents{section}
	[1.25cm] % Left indentation
	{\addvspace{3pt}\sffamily\bfseries} % Spacing and font options for sections
	{\contentslabel[\thecontentslabel]{1.25cm}} % Formatting of numbered sections of this type
	{} % Formatting of numberless sections of this type
	{\hfill\color{black}\thecontentspage} % Formatting of the filler to the right of the heading and the page number

% Subsection text styling
\titlecontents{subsection}
	[1.25cm] % Left indentation
	{\addvspace{1pt}\sffamily\small} % Spacing and font options for subsections
	{\contentslabel[\thecontentslabel]{1.25cm}} % Formatting of numbered sections of this type
	{} % Formatting of numberless sections of this type
	{\ \titlerule*[.5pc]{.}\;\thecontentspage} % Formatting of the filler to the right of the heading and the page number

% Figure text styling
\titlecontents{figure}
	[1.25cm] % Left indentation
	{\addvspace{1pt}\sffamily\small} % Spacing and font options for figures
	{\thecontentslabel\hspace*{1em}} % Formatting of numbered sections of this type
	{} % Formatting of numberless sections of this type
	{\ \titlerule*[.5pc]{.}\;\thecontentspage} % Formatting of the filler to the right of the heading and the page number

% Table text styling
\titlecontents{table}
	[1.25cm] % Left indentation
	{\addvspace{1pt}\sffamily\small} % Spacing and font options for tables
	{\thecontentslabel\hspace*{1em}} % Formatting of numbered sections of this type
	{} % Formatting of numberless sections of this type
	{\ \titlerule*[.5pc]{.}\;\thecontentspage} % Formatting of the filler to the right of the heading and the page number

%----------------------------------------------------------------------------------------
%	MINI TABLE OF CONTENTS IN PART HEADS
%----------------------------------------------------------------------------------------

% Chapter text styling
\titlecontents{lchapter}
	[0em] % Left indentation
	{\addvspace{15pt}\large\sffamily\bfseries} % Spacing and font options for chapters
	{\color{ocre}\contentslabel[\Large\thecontentslabel]{1.25cm}\color{ocre}} % Chapter number
	{}  
	{\color{ocre}\normalsize\sffamily\bfseries\;\titlerule*[.5pc]{.}\;\thecontentspage} % Page number

% Section text styling
\titlecontents{lsection}
	[0em] % Left indentation
	{\sffamily\small} % Spacing and font options for sections
	{\contentslabel[\thecontentslabel]{1.25cm}} % Section number
	{}
	{}

% Subsection text styling (note these aren't shown by default, display them by searchings this file for tocdepth and reading the commented text)
\titlecontents{lsubsection}
	[.5em] % Left indentation
	{\sffamily\footnotesize} % Spacing and font options for subsections
	{\contentslabel[\thecontentslabel]{1.25cm}}
	{}
	{}

%----------------------------------------------------------------------------------------
%	HEADERS AND FOOTERS
%----------------------------------------------------------------------------------------

\usepackage{fancyhdr} % Required for header and footer configuration

\pagestyle{fancy} % Enable the custom headers and footers

\renewcommand{\chaptermark}[1]{\markboth{\sffamily\normalsize\bfseries\chaptername\ \thechapter.\ #1}{}} % Styling for the current chapter in the header
\renewcommand{\sectionmark}[1]{\markright{\sffamily\normalsize\thesection\hspace{5pt}#1}{}} % Styling for the current section in the header

\fancyhf{} % Clear default headers and footers
\fancyhead[LE,RO]{\sffamily\normalsize\thepage} % Styling for the page number in the header
\fancyhead[LO]{\rightmark} % Print the nearest section name on the left side of odd pages
\fancyhead[RE]{\leftmark} % Print the current chapter name on the right side of even pages
%\fancyfoot[C]{\thepage} % Uncomment to include a footer

\renewcommand{\headrulewidth}{0.5pt} % Thickness of the rule under the header

\fancypagestyle{plain}{% Style for when a plain pagestyle is specified
	\fancyhead{}\renewcommand{\headrulewidth}{0pt}%
}

% Removes the header from odd empty pages at the end of chapters
\makeatletter
\renewcommand{\cleardoublepage}{
\clearpage\ifodd\c@page\else
\hbox{}
\vspace*{\fill}
\thispagestyle{empty}
\newpage
\fi}

%----------------------------------------------------------------------------------------
%	THEOREM STYLES
%----------------------------------------------------------------------------------------

\usepackage{amsmath,amsfonts,amssymb,amsthm} % For math equations, theorems, symbols, etc

\newcommand{\intoo}[2]{\mathopen{]}#1\,;#2\mathclose{[}}
\newcommand{\ud}{\mathop{\mathrm{{}d}}\mathopen{}}
\newcommand{\intff}[2]{\mathopen{[}#1\,;#2\mathclose{]}}
\renewcommand{\qedsymbol}{$\blacksquare$}
\newtheorem{notation}{Notation}[chapter]

% Boxed/framed environments
\newtheoremstyle{ocrenumbox}% Theorem style name
{0pt}% Space above
{0pt}% Space below
{\normalfont}% Body font
{}% Indent amount
{\small\bf\sffamily\color{ocre}}% Theorem head font
{\;}% Punctuation after theorem head
{0.25em}% Space after theorem head
{\small\sffamily\color{ocre}\thmname{#1}\nobreakspace\thmnumber{\@ifnotempty{#1}{}\@upn{#2}}% Theorem text (e.g. Theorem 2.1)
\thmnote{\nobreakspace\the\thm@notefont\sffamily\bfseries\color{black}---\nobreakspace#3.}} % Optional theorem note

\newtheoremstyle{blacknumex}% Theorem style name
{5pt}% Space above
{5pt}% Space below
{\normalfont}% Body font
{} % Indent amount
{\small\bf\sffamily}% Theorem head font
{\;}% Punctuation after theorem head
{0.25em}% Space after theorem head
{\small\sffamily{\tiny\ensuremath{\blacksquare}}\nobreakspace\thmname{#1}\nobreakspace\thmnumber{\@ifnotempty{#1}{}\@upn{#2}}% Theorem text (e.g. Theorem 2.1)
\thmnote{\nobreakspace\the\thm@notefont\sffamily\bfseries---\nobreakspace#3.}}% Optional theorem note

\newtheoremstyle{blacknumbox} % Theorem style name
{0pt}% Space above
{0pt}% Space below
{\normalfont}% Body font
{}% Indent amount
{\small\bf\sffamily}% Theorem head font
{\;}% Punctuation after theorem head
{0.25em}% Space after theorem head
{\small\sffamily\thmname{#1}\nobreakspace\thmnumber{\@ifnotempty{#1}{}\@upn{#2}}% Theorem text (e.g. Theorem 2.1)
\thmnote{\nobreakspace\the\thm@notefont\sffamily\bfseries---\nobreakspace#3.}}% Optional theorem note

% Non-boxed/non-framed environments
\newtheoremstyle{ocrenum}% Theorem style name
{5pt}% Space above
{5pt}% Space below
{\normalfont}% Body font
{}% Indent amount
{\small\bf\sffamily\color{ocre}}% Theorem head font
{\;}% Punctuation after theorem head
{0.25em}% Space after theorem head
{\small\sffamily\color{ocre}\thmname{#1}\nobreakspace\thmnumber{\@ifnotempty{#1}{}\@upn{#2}}% Theorem text (e.g. Theorem 2.1)
\thmnote{\nobreakspace\the\thm@notefont\sffamily\bfseries\color{black}---\nobreakspace#3.}} % Optional theorem note
\makeatother

% Defines the theorem text style for each type of theorem to one of the three styles above
\newcounter{dummy} 
\numberwithin{dummy}{section}
\theoremstyle{ocrenumbox}
\newtheorem{theoremeT}[dummy]{Theorem}
\newtheorem{problem}{Problem}[chapter]
\newtheorem{exerciseT}{Exercise}[chapter]
\theoremstyle{blacknumex}
\newtheorem{exampleT}{Example}[chapter]
\theoremstyle{blacknumbox}
\newtheorem{vocabulary}{Vocabulary}[chapter]
\newtheorem{definitionT}{Definition}[section]
\newtheorem{corollaryT}[dummy]{Corollary}
\theoremstyle{ocrenum}
\newtheorem{proposition}[dummy]{Proposition}

%----------------------------------------------------------------------------------------
%	DEFINITION OF COLORED BOXES
%----------------------------------------------------------------------------------------

\RequirePackage[framemethod=default]{mdframed} % Required for creating the theorem, definition, exercise and corollary boxes

% Theorem box
\newmdenv[skipabove=7pt,
skipbelow=7pt,
backgroundcolor=black!5,
linecolor=ocre,
innerleftmargin=5pt,
innerrightmargin=5pt,
innertopmargin=5pt,
leftmargin=0cm,
rightmargin=0cm,
innerbottommargin=5pt]{tBox}

% Exercise box	  
\newmdenv[skipabove=7pt,
skipbelow=7pt,
rightline=false,
leftline=true,
topline=false,
bottomline=false,
backgroundcolor=ocre!10,
linecolor=ocre,
innerleftmargin=5pt,
innerrightmargin=5pt,
innertopmargin=5pt,
innerbottommargin=5pt,
leftmargin=0cm,
rightmargin=0cm,
linewidth=4pt]{eBox}	

% Definition box
\newmdenv[skipabove=7pt,
skipbelow=7pt,
rightline=false,
leftline=true,
topline=false,
bottomline=false,
linecolor=ocre,
innerleftmargin=5pt,
innerrightmargin=5pt,
innertopmargin=0pt,
leftmargin=0cm,
rightmargin=0cm,
linewidth=4pt,
innerbottommargin=0pt]{dBox}	

% Corollary box
\newmdenv[skipabove=7pt,
skipbelow=7pt,
rightline=false,
leftline=true,
topline=false,
bottomline=false,
linecolor=gray,
backgroundcolor=black!5,
innerleftmargin=5pt,
innerrightmargin=5pt,
innertopmargin=5pt,
leftmargin=0cm,
rightmargin=0cm,
linewidth=4pt,
innerbottommargin=5pt]{cBox}

% Creates an environment for each type of theorem and assigns it a theorem text style from the "Theorem Styles" section above and a colored box from above
\newenvironment{theorem}{\begin{tBox}\begin{theoremeT}}{\end{theoremeT}\end{tBox}}
\newenvironment{exercise}{\begin{eBox}\begin{exerciseT}}{\hfill{\color{ocre}\tiny\ensuremath{\blacksquare}}\end{exerciseT}\end{eBox}}				  
\newenvironment{definition}{\begin{dBox}\begin{definitionT}}{\end{definitionT}\end{dBox}}	
\newenvironment{example}{\begin{exampleT}}{\hfill{\tiny\ensuremath{\blacksquare}}\end{exampleT}}		
\newenvironment{corollary}{\begin{cBox}\begin{corollaryT}}{\end{corollaryT}\end{cBox}}	

%----------------------------------------------------------------------------------------
%	REMARK ENVIRONMENT
%----------------------------------------------------------------------------------------

\newenvironment{remark}{\par\vspace{10pt}\small % Vertical white space above the remark and smaller font size
\begin{list}{}{
\leftmargin=35pt % Indentation on the left
\rightmargin=25pt}\item\ignorespaces % Indentation on the right
\makebox[-2.5pt]{\begin{tikzpicture}[overlay]
\node[draw=ocre!60,line width=1pt,circle,fill=ocre!25,font=\sffamily\bfseries,inner sep=2pt,outer sep=0pt] at (-15pt,0pt){\textcolor{ocre}{R}};\end{tikzpicture}} % Orange R in a circle
\advance\baselineskip -1pt}{\end{list}\vskip5pt} % Tighter line spacing and white space after remark

%----------------------------------------------------------------------------------------
%	SECTION NUMBERING IN THE MARGIN
%----------------------------------------------------------------------------------------

\makeatletter
\renewcommand{\@seccntformat}[1]{\llap{\textcolor{ocre}{\csname the#1\endcsname}\hspace{1em}}}                    
\renewcommand{\section}{\@startsection{section}{1}{\z@}
{-4ex \@plus -1ex \@minus -.4ex}
{1ex \@plus.2ex }
{\normalfont\large\sffamily\bfseries}}
\renewcommand{\subsection}{\@startsection {subsection}{2}{\z@}
{-3ex \@plus -0.1ex \@minus -.4ex}
{0.5ex \@plus.2ex }
{\normalfont\sffamily\bfseries}}
\renewcommand{\subsubsection}{\@startsection {subsubsection}{3}{\z@}
{-2ex \@plus -0.1ex \@minus -.2ex}
{.2ex \@plus.2ex }
{\normalfont\small\sffamily\bfseries}}                        
\renewcommand\paragraph{\@startsection{paragraph}{4}{\z@}
{-2ex \@plus-.2ex \@minus .2ex}
{.1ex}
{\normalfont\small\sffamily\bfseries}}

%----------------------------------------------------------------------------------------
%	PART HEADINGS
%----------------------------------------------------------------------------------------

% Numbered part in the table of contents
\newcommand{\@mypartnumtocformat}[2]{%
	\setlength\fboxsep{0pt}%
	\noindent\colorbox{ocre!20}{\strut\parbox[c][.7cm]{\ecart}{\color{ocre!70}\Large\sffamily\bfseries\centering#1}}\hskip\esp\colorbox{ocre!40}{\strut\parbox[c][.7cm]{\linewidth-\ecart-\esp}{\Large\sffamily\centering#2}}%
}

% Unnumbered part in the table of contents
\newcommand{\@myparttocformat}[1]{%
	\setlength\fboxsep{0pt}%
	\noindent\colorbox{ocre!40}{\strut\parbox[c][.7cm]{\linewidth}{\Large\sffamily\centering#1}}%
}

\newlength\esp
\setlength\esp{4pt}
\newlength\ecart
\setlength\ecart{1.2cm-\esp}
\newcommand{\thepartimage}{}%
\newcommand{\partimage}[1]{\renewcommand{\thepartimage}{#1}}%
\def\@part[#1]#2{%
\ifnum \c@secnumdepth >-2\relax%
\refstepcounter{part}%
\addcontentsline{toc}{part}{\texorpdfstring{\protect\@mypartnumtocformat{\thepart}{#1}}{\partname~\thepart\ ---\ #1}}
\else%
\addcontentsline{toc}{part}{\texorpdfstring{\protect\@myparttocformat{#1}}{#1}}%
\fi%
\startcontents%
\markboth{}{}%
{\thispagestyle{empty}%
\begin{tikzpicture}[remember picture,overlay]%
\node at (current page.north west){\begin{tikzpicture}[remember picture,overlay]%	
\fill[ocre!20](0cm,0cm) rectangle (\paperwidth,-\paperheight);
\node[anchor=north] at (4cm,-3.25cm){\color{ocre!40}\fontsize{220}{100}\sffamily\bfseries\thepart}; 
\node[anchor=south east] at (\paperwidth-1cm,-\paperheight+1cm){\parbox[t][][t]{8.5cm}{
\printcontents{l}{0}{\setcounter{tocdepth}{1}}% The depth to which the Part mini table of contents displays headings; 0 for chapters only, 1 for chapters and sections and 2 for chapters, sections and subsections
}};
\node[anchor=north east] at (\paperwidth-1.5cm,-3.25cm){\parbox[t][][t]{15cm}{\strut\raggedleft\color{white}\fontsize{30}{30}\sffamily\bfseries#2}};
\end{tikzpicture}};
\end{tikzpicture}}%
\@endpart}
\def\@spart#1{%
\startcontents%
\phantomsection
{\thispagestyle{empty}%
\begin{tikzpicture}[remember picture,overlay]%
\node at (current page.north west){\begin{tikzpicture}[remember picture,overlay]%	
\fill[ocre!20](0cm,0cm) rectangle (\paperwidth,-\paperheight);
\node[anchor=north east] at (\paperwidth-1.5cm,-3.25cm){\parbox[t][][t]{15cm}{\strut\raggedleft\color{white}\fontsize{30}{30}\sffamily\bfseries#1}};
\end{tikzpicture}};
\end{tikzpicture}}
\addcontentsline{toc}{part}{\texorpdfstring{%
\setlength\fboxsep{0pt}%
\noindent\protect\colorbox{ocre!40}{\strut\protect\parbox[c][.7cm]{\linewidth}{\Large\sffamily\protect\centering #1\quad\mbox{}}}}{#1}}%
\@endpart}
\def\@endpart{\vfil\newpage
\if@twoside
\if@openright
\null
\thispagestyle{empty}%
\newpage
\fi
\fi
\if@tempswa
\twocolumn
\fi}

%----------------------------------------------------------------------------------------
%	CHAPTER HEADINGS
%----------------------------------------------------------------------------------------

% A switch to conditionally include a picture, implemented by Christian Hupfer
\newif\ifusechapterimage
\usechapterimagetrue
\newcommand{\thechapterimage}{}%
\newcommand{\chapterimage}[1]{\ifusechapterimage\renewcommand{\thechapterimage}{#1}\fi}%
\newcommand{\autodot}{.}
\def\@makechapterhead#1{%
{\parindent \z@ \raggedright \normalfont
\ifnum \c@secnumdepth >\m@ne
\if@mainmatter
\begin{tikzpicture}[remember picture,overlay]
\node at (current page.north west)
{\begin{tikzpicture}[remember picture,overlay]
\node[anchor=north west,inner sep=0pt] at (0,0) {\ifusechapterimage\includegraphics[width=\paperwidth]{\thechapterimage}\fi};
\draw[anchor=west] (\Gm@lmargin,-9cm) node [line width=2pt,rounded corners=15pt,draw=ocre,fill=white,fill opacity=0.5,inner sep=15pt]{\strut\makebox[22cm]{}};
\draw[anchor=west] (\Gm@lmargin+.3cm,-9cm) node {\huge\sffamily\bfseries\color{black}\thechapter\autodot~#1\strut};
\end{tikzpicture}};
\end{tikzpicture}
\else
\begin{tikzpicture}[remember picture,overlay]
\node at (current page.north west)
{\begin{tikzpicture}[remember picture,overlay]
\node[anchor=north west,inner sep=0pt] at (0,0) {\ifusechapterimage\includegraphics[width=\paperwidth]{\thechapterimage}\fi};
\draw[anchor=west] (\Gm@lmargin,-9cm) node [line width=2pt,rounded corners=15pt,draw=ocre,fill=white,fill opacity=0.5,inner sep=15pt]{\strut\makebox[22cm]{}};
\draw[anchor=west] (\Gm@lmargin+.3cm,-9cm) node {\huge\sffamily\bfseries\color{black}#1\strut};
\end{tikzpicture}};
\end{tikzpicture}
\fi\fi\par\vspace*{270\p@}}}

%-------------------------------------------

\def\@makeschapterhead#1{%
\begin{tikzpicture}[remember picture,overlay]
\node at (current page.north west)
{\begin{tikzpicture}[remember picture,overlay]
\node[anchor=north west,inner sep=0pt] at (0,0) {\ifusechapterimage\includegraphics[width=\paperwidth]{\thechapterimage}\fi};
\draw[anchor=west] (\Gm@lmargin,-9cm) node [line width=2pt,rounded corners=15pt,draw=ocre,fill=white,fill opacity=0.5,inner sep=15pt]{\strut\makebox[22cm]{}};
\draw[anchor=west] (\Gm@lmargin+.3cm,-9cm) node {\huge\sffamily\bfseries\color{black}#1\strut};
\end{tikzpicture}};
\end{tikzpicture}
\par\vspace*{270\p@}}
\makeatother

%----------------------------------------------------------------------------------------
%	LINKS
%----------------------------------------------------------------------------------------

\usepackage{hyperref}
\hypersetup{hidelinks,backref=true,pagebackref=true,hyperindex=true,colorlinks=false,breaklinks=true,urlcolor=ocre,bookmarks=true,bookmarksopen=false}

\usepackage{bookmark}
\bookmarksetup{
open,
numbered,
addtohook={%
\ifnum\bookmarkget{level}=0 % chapter
\bookmarksetup{bold}%
\fi
\ifnum\bookmarkget{level}=-1 % part
\bookmarksetup{color=ocre,bold}%
\fi
}
}
 % Insert the commands.tex file which contains the majority of the structure behind the template

% Packages
\usepackage[utf8]{inputenc}
\usepackage{amsthm}
\usepackage{amsmath}
\usepackage{amssymb}
\usepackage{romannum}
\usepackage{textcomp}
\usepackage{titlesec}
\usepackage{array}
\usepackage{booktabs}

% Counters
\newcounter{example}
\newcounter{claim}
\newcounter{solution}
\setcounter{secnumdepth}{4}

\theoremstyle{claim}
\newtheorem{claim}{Claim}[section]

% Commands

\newcommand\Example{%
  \stepcounter{example}%
  \textbf{Example \theexample.}~%
  \setcounter{solution}{0}%
}

% The Solution is used when only
% 1 solution exists
\newcommand\TheSolution{%
  \textbf{Solution:}\\%
}

% If more than 1 solution exists to a
% specific problem/ example use 
% a solution.
\newcommand\ASolution{%
  \stepcounter{solution}%
  \textbf{Solution \thesolution:}\\%
}

\title{Abstract Structures 333}
\begin{document}
\maketitle

% Flush left to keep starting point
% of all lines flush (Design pref.)

\section{Equivalence Relations}
\begin{definition}{Equivalence Relation}
An equivalence relation, denoted by the symbol \char`\~ , on a set $\Huge S$ is a set $\Huge R$\footnote{Need not be unique}
of ordered pairs (a, b) $\in$ S x S such that:

\begin{enumerate}
  \item (a, a) $\in$ R $\forall$ a $\in$ S
  \item (a, b) $\in$ R implies (b, a) $\in$ R $\forall$ a, b $\in$ S
  \item (a, b), (b, c) $\in$ R implies (a, c) $\in$ R $\forall$ a, b, c $\in$ S
\end{enumerate}
\end{definition}\\
Every Equivalence Relation $\Huge R$, imposes a partition on $\Huge S$\\\\
\Example Define $\mathbb{Z}$ in the following way:\\
Fix n $\in$ $\mathbb{Z+}$ 
$a \equiv b \mod n \iff n \mid (a-b)$\\\\
Show that the example above is an equivalence relation

\TheSolution 
\begin{proof}{}
We must prove the following 3 properties
\begin{enumerate}
  \item $\bold{Reflexive}$ [ ($\romannum{1}$) in the def of $ER$]
  \begin{itemize}
     \item Thought of as: An element a is always related (\char`\~) to itself.
   \end{itemize}
   We are trying to prove that $a \equiv a \mod n$. We can start by rewriting this congruence as $n \mid (a-a)$ by def of congruence. This leaves us with $n \mid (0)$ which is true for all $n > 0$. Since n be def is fixed in $\mathbb{Z+}$, this congruence will always hold.
  \item $\bold{Symmetric}$ [ ($\romannum{2}$) in the def of $ER$]
    \begin{itemize}
     \item Thought of as: Given $(a, b)$ is valid, we can show $(b, c)$ is valid.
    \end{itemize}
   Since we are given $(a, b)$ is valid, we can write $a \equiv b \mod n$ or $n \mid a-b$. We must show that $b \equiv a \mod n$ or $n \mid (b-a)$.
   We can rewrite $n \mid (b-a)$ as $-1 * n \mid (a-b)$. Since we know $n \mid (a-b)$ from out given, we know that this division holds true and therefore $n \mid (b-a)$ as well. 
  \item $\bold{Transitive}$ [ ($\romannum{3}$) in the def of $ER$]
    \begin{itemize}
     \item Thought of as: Given $a \char`\~ b$ and $b \char`\~ c$ we must show $a \char`\~ c$.
   \end{itemize}
   We can write the congruence as 2 linear equation.
   \begin{itemize}
     \item $nk = a - b$
     \item $nl = a - c$
   \end{itemize}
   Rearranging we get: $n(k+l) = a - c$ which can be rewritten as $n \mid (a-c)$
\end{enumerate}
\end{proof}
Now that we have proved that a congruence is an $ER$ on $S = \mathbb{Z}$ we would like to see what affect it has on $\mathbb{Z}$. $ie:$ What is $a \char`\~ b$ / what partition does it impose.
\newline
\newline
\Example Take $n = 5$, given the following values for $a$ which values in $\mathbb{Z}$ satisfy the congruence $a \equiv b \mod n$ and is the resulting set equal to $\mathbb{Z}$?
\TheSolution 
\begin{itemize}
     \item $a = 0 = \{\pm{0}, \pm{5}, \pm{10}\dots\}$
     \item $a = 1 = \{\pm{1}, \pm{6}, \pm{5k+1}\dots\}$
     \item $a = 2$ = \{\pm{2}, \pm{7}, \pm{5k+2}\dots\}$
\end{itemize}
No sets by themselves equal $\mathbb{Z}$ but together (along with $a = 3, 4$) they do.
We can see that it appears that a congruence will always split the set $\mathbb{Z}$ into n partitions.
\newline
\begin{definition}{Partition of a set}

A partition of a set S is a collection of \textbf{non-empty, disjoint} subsets $\{s_{0}, s_{1}, \dots\}$ such that (st) $\bigcup\limits_{i=1}^{\infty} S_{i} = S$

\end{definition}
\begin{theorem}{The equivalence classes of a set S under \char`\~ form a partition of  $\Huge S$}
\end{theorem}
\begin{proof}{}
We need to show that given \char`\~, we are left with a collection of \textbf{disjoint} subsets who's union is  $\Huge S$.\\
Let a \char`\~   $\Huge S$. We know $a$ is in  its own set because a \char`\~ $a$. So $\forall$ $a$ $\in$  $\Huge S$ the set containing a is \textbf{non-empty}. If we do this for all $a$ $\in$ $\Huge S$ then the union of those sets is S. So we need only show that these sets are \textbf{disjoint}.
\end{proof}

\Example Let  $\Huge S$ = $\mathbb{Z}x\mathbb{Z}$ [(a, b) a,b $\in$ $\mathbb{Z}$]
Define $\char`\~$ on  $\Huge S$ by $(a, b)$ $\char`\~$ (c, d) \iff $ad = bc$
\begin{enumerate}
  \item Prove $\char`\~$ is an ER
  \item What partition of $\mathbb{R}x\mathbb{R}$ does this impose
\end{enumerate}\\
\TheSolution 
\begin{proof}
If ER, 3 properties must hold:
\begin{enumerate}
  \item Reflexive: $(a, b)$ $\char`\~$ $(a, b)$ $\Longrightarrow$ ab = ab which is true.
  \item Symmetric: Given $(a, b$) $\char`\~$ $(c, d)$ we can show ($c, d)$ $\char`\~$ $(a, b)$. $(a, b)$ $\char`\~$ (c, d) $\Longrightarrow$ $ad = bc$, $(c, d)$ $\char`\~$ (a, b) $\Longrightarrow$ $cb = da$. Since we are in the realm of $\mathbb{R}$ we can rearrange to $bc = ad$ which is equal to $ad = bc$.
  \item Transitive: We must show that if $(a, b)$ $\char`\~$ $(c, d)$ and $(c, d)$ $\char`\~$ $(e, f)$ then $(a, b)$ $\char`\~$ $(e, f)$. We can write it as follows $ad = bc$ and $cf = de$  the $af = be$. What follows is
  \begin{gather*} 
  adcf = bcde   \\
  ace = bce     \\
  af = be       \\
  \end{gather*} 
\end{enumerate}
To find the partition we may start by plugging in random values.\\
\begin{gather*} 
(1, 1)  \\
1d = 1c \\
d = c   \\
= \{(1, 1), (2, 2), \dots, (n, n)\}
\end{gather*}
\begin{gather*} 
(1, 2)\\
d = 2c\\
= \{(1, 2), (2, 4), \dots, (n, 2n)\}\\
\vdots\\
\infty
\end{gather*} 
This partition forms all rational numbers. The first set represents $\frac{1}{1}$ or $1$, the second represents $\frac{1}{2}$ $\dots$ $\infty$
\end{proof}
\Example Let S = $\mathbb{R}-\{0\}$\\Define a $\char`\~ b \leftrightarrow ab > 0$\\What partition does that make on $\mathbb{R}$\newline\\
\TheSolution By plugging in we see we get 2 sets.
\begin{enumerate}
  \item \{1, 2, \dots, n\} = All positive integers
  \item \{-n, -n-1, \dots, -1\} = All negative integers
\end{enumerate}
\begin{theorem}{Division Algorithm}\\
Let D $\in$ $\mathbb{Z+}$, a $\in$ $\mathbb{Z}$, $\exists !$q, r s.t. $a = dq + r$ when $0 < r \leq d$
\end{theorem}

\Example a = 100, d = 7\newline\\
\TheSolution
\begin{gather*} 
100 = 7q + r = 7(14) + 2\\
7 = 2q + r = 2(3) + 1   \\
2 = 1q + r = 1(2) + 0   \\
\end{gather*} 
So, 1 would be the GCD.

\section{Groups}

\begin{definition}{Binary Operation}
We define a binary operation on set S is a function from SxS $\longrightarrow$ S\\
ie: Takes a pair of elements in S and sends them to another element in S
\end{definition}
\Example Let S = $\mathbb{Z}$, with bin-op (+)\\
a + b = c\\
3 + 5 = 8\\
3 $\in$ $\mathbb{Z}$, 5 $\in$ $\mathbb{Z}$, 8 $\in$ $\mathbb{Z}$.\\

\begin{definition}{Let S be a set w/ bin-op $*$\footnote{$*$ denotes any bin-op}}\\
If $\forall$ a, b $\in$ S, a + b $\in$ S we say S is closed (under $*$)
\end{definition}

\Example\\
($M_{22}$, $\cdot$) is closed \\
($\mathbb{R}$, $\div$) is not closed\\

\begin{definition}{Let G be a set closed under bin-op *}
G is a group if the following hold:
\begin{enumerate}
  \item Associative: $\forall$ a, b, c $\in$ G we have $(a * b) * c = a * (b * c)$
  \item $\exists$ an Identity in G s.t. $\forall$ a $\in$ G we have $(e * a) = (a * e) = a$
  \item ($\forall$ a $\in$ G)$\exists a^{-1}$ s.t. $a * a\prime = a^{-1} * a = e$
\end{enumerate}
\end{definition}
\Example $\mathbb{Z}_{n}$ = the group \{0, 1, 2, $\dots$, n-1\} under addition mod n.\\
What is addition mod n?\\

\TheSolution 
For a, b $\in \mathbb{Z}_{n}$:\\
if $a + b < n$, $a + b = a + b$ \\
if $a + b \geq n$, $a + b = a + b - n$\\

\begin{enumerate}
  \item Associative: We are dealing with integers so associativity holds (inherited)
  \item $\exists$ an Identity: The identity is 0 ($e = 0$) 
  \item ($\forall$ a $\in$ G)$\exists a^{-1}$: The inverse is $n - a$
\end{enumerate}

\paragraph{Common Groups}
\begin{itemize}
  \item ($\mathbb{Z}$, +)
  \item ($\mathbb{R}$, +)
  \item ($\mathbb{C}$, +)
  \item ($GL_{2R}$, *)
\end{itemize}

\begin{proof} of  ($GL_{2R}$, *)\\
  We know from linear algebra that $det(AB) = det(A)det(B)$\\
  We also know that the identity 2x2 matrix is:
  \[
M_{2x2}=
  \begin{bmatrix}
    1 & 0  \\
    0 & 1
  \end{bmatrix}
\]\\
Additionally, we are able to inherit associativity from general matrices. 
This leaves inverse.\\
We prove inverse as follows:
A^{-1} = \frac{1}{det(A)} 
\[
    \begin{bmatrix}
      d & -b  \\
      -c & a
    \end{bmatrix}
\]
\end{proof}\\

\begin{definition}{Order of a  group}\\
The order of a group, G, denoted $| G |$ is the number of elements in G as a set\\
If set G has a finite number of elements we say G is a finite group. If G has an infinite number of elements we say G is an infinite group.
\end{definition}

\begin{definition}{Abelian Groups}\\
If a group is commutative, we say it is Abelian. If not, we say its not-Abelian. More formally, $\forall (a, b) \in G$, We call $G$ Abelian \iff $ab = ba$
\end{definition}

\begin{definition}{Cayley Table}\\
A cayley table is a way to describe the structure of a finite group.\\
Properties that may be derived from a cayley table are:
\begin{itemize}
  \item If the table is reflect-able, the group is Abelian 
  \item Every element appears in each row/column
  \item Easily find the identity (The row/column which entries is equal to the input)
\end{itemize}
\end{definition}

\Example Write the Cayley Table for $\mathbb{Z}_{3}$\\

\TheSolution 
 \begin{center}
\setlength\extrarowheight{3pt}
\begin{tabular}{c | c c c c c}
    $\mathbb{Z}_{3}$ & 0 & 1 & 2  \\
    \cline{1-4}
    0 & 0 & 1 & 2  \\
    1 & 1 & 2 & 0  \\
    2 & 2 & 0 & 1  \\
\end{tabular}
\end{center}

\Example Write the Cayley Table for $|G|=3$\\

\TheSolution 
 \begin{center}

\setlength\extrarowheight{3pt}
\begin{tabular}{c | c c c c c}
    $|G|=3$ & e & a & b  \\
    \cline{1-4}
    e & e & a & b  \\
    a & a & b & e  \\
    b & b & e & a  \\
\end{tabular}
\end{center}
Notice that this the second table above was forced. Meaning, no other configuration of $e,a,b$ could have been entered into the table and the table maintain all group properties.\\\\
We see from this that there is only 1 group with order 3. Even though we may label that group with different elements, the underlying groups are all the same.

\begin{claim}{}
$\exists$! 2 groups of order 4 ($|G|=4$)
\end{claim}

\begin{center}
\setlength\extrarowheight{3pt}
\begin{tabular}{c | c c c c c c}
    $\mathbb{Z}_{4}$ & 0 & 1 & 2 & 3  \\
    \cline{1-5}
    0 & 0 & 1 & 2 & 3  \\
    1 & 1 & 2 & 3 & 0 	\\
    2 & 2 & 3 & 0 & 1	\\
    3 & 3 & 0 & 1 & 2	\\
\end{tabular}
\end{center}

\begin{center}
\setlength\extrarowheight{3pt}
\begin{tabular}{c | c c c c c c}
    $\mathbb{Z}_{2x2}$ &(0,0) & (0,1) & (1,0) & (1,1)  \\
    \cline{1-5}
    (0,0) & (0,0) & (0,1) & (1,0) & (1,1)  	\\
    (0,1) & (0,1) & (0,0) & (1,1) & (1,0) 	\\
    (1,0) & (1,0) & (1,1) & (0,0) & (0,1)		\\
    (1,1) & (1,1) & (1,0) & (0,1) & (0,0)		\\
\end{tabular}
\end{center}\\
Any other groups of order 4 will have a bijection to either $\mathbb{Z}_{4}$ or $\mathbb{Z}_{2x2}$\\
Here is an example of one of those:

\Example Let G = \{1, -1, i, -i\} under *\\
 
\begin{center}
\setlength\extrarowheight{3pt}
\begin{tabular}{c | c c c c c c}
    $\mathbb{G}_{4}$ & 1 & -1 & i & -i  \\
    \cline{1-5}
    $1$&$1$&$-1$&$ i$&$-i$  	l\\
    $-1$&$-1$&$1$&$-i$&$i$	\\
    $i$&$i$&$-i$&$-1$& $1$	\\
    $-i$&$-i$&$i$&$1$&$-1$	\\
\end{tabular}
\end{center}

\begin{claim}{}
If groups are structurally identical, then, you can find a bijection $\phi$(G_{1}) =  G_{2}
\end{claim}\\

\begin{definition}{Order of an element}\\
Let $G$ be a group with g $\in$ $G$. The order of g (referred to as the order of the element) is the smallest positive integer n s.t. $g^{n} = e$ where e is the identify element of $G$. 
\end{definition}

\begin{definition}{Cyclic Group}\\
Let $G$ be a group with order $n$. We say $G$ is cyclic if $\exists$ g $\in G$ s.t $|g|=n$
\end{definition}

\begin{theorem}{Let a $\in  \mathbb{Z}_{n}$, then $|a| = \frac{n}{(a,n)}$\end{theorem}

\begin{proof}{$|a|$ is the smallest positive integer $k$ s.t. $ka  \equiv 0 (n)$\\i.e $ln-ka = 0$\\Solve for k using linear diophantine equation}

\begin{claim}{
\mathbb{Z}_{n}$ is a cyclic group} 
\end{claim}\\

\begin{proof}{

We know  $\mathbb{Z}_{n}$ is cyclic if $\exists$ some a s.t. $|a| = n$ where $n = |G|$ by the proof above (Them. 3), $|a| = \frac{n}{(a,n)}$$\therefore$ $(a, n) = 1$. 

To prove that $\mathbb{Z}_{n}$ is cyclic we must show that $\exists$ an a such that $(a, n) = 1$\\We will choose $n-1$ as our $a$ giving us $(n-1, n) = 1$ which is always true}

\end{proof}

\begin{definition}{Group Generator}\\
Let $G$ be a cyclic group of order $n$.\\
If $a\inG$ has order $n$, we call $a$ a generator of $G$ and we write $<a> = G$ . We say "the group generated by a"
\end{definition}

\begin{claim}{
Every element $a \in \mathbb{Z}_{n}$ that is relatively prime to $n$ ($gcd(a, n) = 1$) is a generator} 
\end{claim}\\

\begin{definition}{$\mathbb{U}_{n}$}\\
$\mathbb{U}_{n}$ is defined as all elements in $\mathbb{Z}_{n}$ that are relatively prime to $n$ ($gcd(a, n) = 1$)
\end{definition}

\begin{theorem}{All cyclic groups are Abelian}\end{theorem}
\begin{proof}{

We must show that in cyclic group G, every 2 elements commute.

Since G is cyclic $\exists$ $a$ $\in$ $G$ $s.t$ $<a> = G$.

Take 2 $distinct$ elements $\in$ $G$, call them $g_{1}$ and  $g_{2}$.

\begin{center}
$g_{1} = a^{x}$ for some x

$g_{2} = a^{y}$ for some y
\end{center}

We can write $g_{1}g_{2} = g_{2}g_{1} = a^{x}a^{y} = a^{y}a^{x} = a^{x+y} = a^{y+x}$

This final step ($a^{x+y} = a^{y+x}$) is valid because $x$ and $y$ are integers!


}
\end{proof}
\end{document}
